The project will involve a University of Delaware graduate student, and it will constitute the core of their thesis.  We will collaborate closely with a Johns Hopkins graduate student in the first year of the project whose thesis is also on light echoes and transient identification in light echoes.  We will involve one undergraduate each year on the project. 

The broader impact of this proposal will focus on machine learning and data science education of undergraduate and graduate students at the University of Delaware (UD) and John Hopkins University. 
This skill are becoming indispensable in an increasingly technological world.  As part of her faculty responsibility toward the Department of Physics and Astronomy (DPA) at the UD, PI Bianco is developing data science and machine learning curriculum for physicists and astronomers (two classes have been developed this year: Data Science for Physical Scientists\footnote{\url{https://github.com/fedhere/dsps}} which is offered at the undergraduate and graduate level, and Machine Learning for Time Series Analysis which will be offered at the graduate level next semester). 
The UD graduate student and the undergraduates involved in this proposal will be encouraged to take these classes, which are project based designed to provide a hands-on learning environment ideal for developing data-driven research skills. 

Projects related to this proposal will offer a research opportunity for undergraduate students that are interested in the new 4+1 Physics and Data Science program at UD. UD launched a Master Degree in Data Science\footnote{https://www.msds.udel.edu/} in 2018 and will expand it to include a 4+1 joint degree with the DPA next year (the 4+1 is under academic review).

Another important opportunity  will be available to all student involved in this proposal (at UD and Hopkins), and to all UD students, through a UD hackathon program that PI Bianco and CoI Brockmeier will start at the UD Data Science Institute.  This program will organize UD-wide hackathons for all UD students at least twice a year.

Hackathon and data dives are prime examples of active learning activities.  We will organize hackathons twice yearly as a weekend long event.  Students will work in self-assembled groups on projects of their choice and produce a deliverable, which they will present at the end of the event in front of judges and a small audience.  Open to all interested undergraduates, the hackathons will have a variety of topics, both proposed by the UD faculty. LSST-related projects will be made available at each session, including projects related to the research supported by this proposal. Following the significant success of the recent LSST Transient and Variable Stars Collaboration & Stars Milky way, Local Volume Science Collaboration joint hackathon\footnote{https://lsst-tvssc.github.io/TVS-SMWLV2019/}, whenever possible these events will overlap with LSST Science Collaboration meetings, to broaden the set of experts that can propose pitches as well as mentor working students, also enabling the prototyping of solutions for LSST data analysis. However, all sorts of projects will be welcome! including projects in the social sciences (Pi Bianco is already engaged in public policy data driven research as testified by her affiliation with the Biden School of Public Policy and Administration, and Collaborator Dobler, also a member of both departments, has focused his research and education portfolio on evidence-based public policy studies).  Several colleagues across the UD departments and colleges have expressed interest and desired to contribute to these events already! This is an ideal venue for building superskills, especially collaborative and communication skills, in our students. LSST has supported hackathons in various environment as a way to jump-start LSST related research and PI Bianco has been involved in the leadership and organization of many of these events\footnote{http://fbb.space/LSSTHackathonCCA/WPgallery.html, https://lsst-tvssc.github.io/TVS-SMWLV2019/}.

A typical hackathon event starts with project pitches.  Students are encouraged to present project ideas, but projects are also available and presented by the moderating faculty and postdoc.  Occasionally, external agencies or companies are invited to pitch projects as well.  In that case, they may provide prizes to the winning teams.  A pitch would include a problem statement, the desired deliverable, and the skills required for the project.
Groups self-assemble to work on projects based on interest and required expertise and work continuously, including on the preparation of a 5-10 minutes slides presentation that concludes the event.  During the hackathon, mentors are available to help and advise the students on methodology, and a moderator makes sure the tight timeline is respected, including containing the initial pitches and team assembling time, and assuring that students prepare the slides timely for the presentation. 
Project results are presented by each team.  Feedback is given separately on the presentation, by an assigned "critique", and on the content by the judges, that have priority in asking questions and giving feedback, then time permitting additional questions are asked by the audience.  Projects are ranked based on their success in four main areas: 
\begin{itemize}
\item the quality of the presentation
\item the quality of the deliverable
\item how well the deliverable complies with the request formulated in the pitch 
\item the application of adequate and appropriate methodologies.
\end{itemize}

In these events, the students see the full life-cycle of a data-driven project, from conception to presentation.  This is a rare opportunity for them since in course-based instruction, even in project-based classes, generally the research is guided and extended over time, often resulting in a dispersive, as opposed to immersive, experience. The postdoc hired under this project will serve as a mentor and organizer, together with PI and CoI Bianco and Brockmeier, and as a judge at the final presentation showdown (see Postdoc Mentoring Plan)
