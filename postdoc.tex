This proposal will support 50\% of the effort of a postdoctoral researchers.  The postdoctoral researcher mentoring plan for this program will be overseen by the PI Dr. Bianco (Assistant Professor at the UD, Physics and Astronomy - DPA) and CoI Dr. Brockmeier (Assistant Professor at the UD, Electrical Engineering).  The different domains of Prof. Bianco and Brockmeier will provide a complete multi-faceted point of view in reviewing and advising the postdocs on science and job-success related matters. 
The postdoctoral researcher will be recruited with a solicitation issued by PI Bianco on November 2019.  The PIs will create a Postdoctoral Selection Committee including CoI Brockmeier and two senior faculty members within the Department of Physica and Astronomy at UD to review the applications.

To assure that a diverse pull of candidates is reached, PI Bianco has has posted the job announcement on several venues that target diverse scholars (e.g. Diverse Issues in Higher Ed. and Hispanic Outlook).  If a  scholar from an under-represented minority is hired, they will benefit from the commitment of UD and specifically the renewed commitment of the DPA to create an equitable and supportive environment with a new Diversity and Inclusion Committee\footnote{http://bit.ly/UDDPADIcommittee}, created and chaired by PI Bianco at the UD DPA.

The PI and CoI will offer, as part of the mentoring program: 
1) Training in preparation of grant proposals: Being able to effectively and consistently find funding is a compulsory skill required of successful researchers.  The postdoctoral researcher should be coached in and encouraged to submit proposals (e.g. postdoctoral fellowship, HST proposals) and encouraged to serve on at least one review panel to gain a deeper understanding of the review process. 
2) Publications and presentations: The mentors will support (through this grant and her start-up funds) travel support to assure the postdocs can present their work at conferences and publish in high impact journals.  The postdocs should travel to the LSST annual Community Workshop, international LSST meetings (e.g. LSST@Europe, on a bi-annual basis), and topical AI meetings. 
3) Mentoring: The postdoc will have opportunities to interact with and mentor students. They will mentor the graduate and undergraduate students working with PI Bianco and CoI Rest, but the PI will also involve the postdoc in hackathons as mentor, both in LSST meetings and at the UD Data Science Institute, where PI Bianco and CoI Brockmeier are developing a hackathon program of 2 events each year (PI Bianco has experience developing such programs at NYU Center for Urban Science and Progress and elsewhere, see biosketch).  The postdoctoral researcher will be involved in  planning and preparing the hackathons and will serve as mentors during the hackathon sessions and as judges at the hackathon presentations.
4) Career Counselling: A likely outcome for the postdoc is  employment outside of academia: in recent years the number of graduating PhD physics students ranged between 1500 and 2000, while the number of  faculty hires ranges between 200 and 300\footnote{\url{https://www.aip.org/sites/default/files/statistics/physics-trends/fall16-phdsubfield-p1.pdf}}\footnote{\url{https://www.aip.org/statistics/physics-trends/number-faculty-hired-physics-departments}}.  The postdoctoral researcher will data science learning skills that are crucial in the tech job market and will be encouraged to continually discuss their career options with their mentors, peers, and collaborators, so that they can successfully gauge the appropriate skills required for their chosen career path.
Assessment: The personal progress of the postdoctoral researchers will be measured against the expectations laid out in this mentoring plan, i.e.: number of papers and proposals submitted, the number of seminars given, quality of lectures, and  effectiveness of their mentoring (student success).  Every 4 months the postdoctoral research will meet separately with PI Bianco and CoI Brockmeier so that they will be able to raise concerns about mentoring.  Dr. Bianco and Dr. Brockmeier will share a report with the postdoctoral research every 4 months.  
