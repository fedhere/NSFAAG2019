REFERENCES
[1] Graur, Or, Federica B. Bianco, and Maryam Modjaz. (2015) “A Unified Explanation for the Supernova Rate-Galaxy Mass Dependence Based on Supernovae Detected in Sloan Galaxy Spectra.” Monthly Notices of the Royal Astronomical Society 450.1 : 905–925. 
[2] Abbott, B. P. et al. (2016) “Observation of gravitational waves from a binary black hole merger.” Physics Review Letters 116, 061102 
[3] Rest, A., [with F. B. Bianco] et al. (2012). “Light echoes reveal an unexpectedly cool η Carinae during its nineteenth-century Great Eruption” Nature, Volume 482, Issue 7385, pp. 375-378  http://adsabs.harvard.edu/abs/2012Natur.482..375R
[4] Masci, Frank J. et al. (2018) “The Zwicky Transient Facility: Data Processing, Products, and Archive.” Publications of the Astronomical Society of the Pacific 131.995 : 018003. 
[5] Ivezi’c [w F.B. Bianco] et al. (2008) “LSST: from Science Drivers to Reference Design and Anticipated Data Products.” arXiv:0805.2366. https://arxiv.org/abs/0805.2366
[6] LSST Science Collaborations,  LSST Project, Abell, P., A. et al. (2009).  LSST Science Book: version 2.0. LSST Corp., https://www.lsst.org/scientists/scibook 
[7] Rest, A., et al.  (2005) “Light echoes from ancient supernovae in the Large Magellanic Cloud”. Nature, Volume 438, Issue 7071, pp. 1132-1134 (2005) http://adsabs.harvard.edu/abs/2005Natur.438.1132R
[8] Rest, A. et al.  (2011) “DIRECT CONFIRMATION OF THE ASYMMETRY OF THE CAS A SUPERNOVA WITH LIGHT ECHOES.” The Astrophysical Journal 732.1
[9] Finn, Kieran; Bianco, Federica B.; Modjaz, Maryam; Liu, Yu-Qian; Rest, Armin, (2016) “Comparison of Diversity of Type IIb Supernovae with Asymmetry in Cassiopeia A Using Light Echoes”. The Astrophysical Journal, Volume 830, Issue 2, article id. 73, 11 pp.  http://adsabs.harvard.edu/abs/2016ApJ...830...73F
[10] Rest, Armin et al.  (2008). “Scattered-Light Echoes from the Historical Galactic Supernovae Cassiopeia A and Tycho (SN 1572).”
[11] Bianco, Federica; et al. (2019); “Better support for collaborations preparing for large-scale projects: the case study of the LSST Science Collaborations” Astro2020: Decadal Survey on Astronomy and Astrophysics, APC white papers, no. 185; Bulletin of the American Astronomical Society, Vol. 51, Issue 7, id. 185 
[12] F. B. Bianco et al. (2009) “A SEARCH FOR OCCULTATIONS OF BRIGHT STARS BY SMALL KUIPER BELT OBJECTS USING MEGACAM ON THE MMT”. The American Astronomical Society. The Astronomical Journal, Volume 138, Number 2. https://iopscience.iop.org/article/10.1088/0004-6256/138/2/568
[13] Bianco, F. B., Modjaz, M., Hicken, M., Friedman, A., Kirshner, R. P., Bloom, J. S., Challis, P., Marion, G. H., Wood-Vasey, W. M., & Rest, A. (2014), “Multi-color Optical and Near-infrared Light Curves of 64 Stripped-envelope Core-Collapse Supernovae”. The Astrophysical Journal, 213, 19.
[14] Couderc, P. (1939). "Les Auréoles Lumineuses des Novae". Annales d'Astrophysique. 2: 271–302. 
[15] Goodfellow I. J., Pouget-Abadie J., Mirza M., Xu B., WardeFarley D., Ozair S., Courville A., Bengio Y. (2014), “Generative Adversarial Networks” Neural Information Processing Systems Conference.
[16] Reiman, D & Gohre B (2018), “Deblending galaxy superpositions with branched GANs”, https://ui.adsabs.harvard.edu/abs/2019MNRAS.485.2617R/abstract
[17] B. Steers, J. Kastelan, C. C. Tsai, F.B. Bianco; G. Dobler (2019). “Detection of polluting plumes ejected from NYC buildings”. DOI: https://doi.org/10.22541/au.155534109.95420301
[18] Ashish Shrivastava, Tomas Pfister, Oncel Tuzel, Josh Susskind, Wenda Wang, Russ Webb (2016) “Learning from Simulated and Unsupervised Images through Adversarial Training” IEEE Conference on Computer Vision and Pattern Recognition (CVPR) 
[19] N. Smith, J. E. Andrews, A. Rest, F. B. Bianco, et al. (2018) “Light echoes from the plateau in Eta Carinae’s 13 federica b. bianco, 2019 - Publications PUBLICATION LIST federica b. bianco Great Eruption reveal a two-stage shock-powered event”. In: MNRAS 480 , pp. 1466–1498. doi: 10.1093/mnras/sty1500. arXiv: 1808.00992 [astro-ph.SR]. 
[20] N. Smith, et al. [w F. B. Bianco] (2018). “Exceptionally fast ejecta seen in light echoes of Eta Carinae’s Great Eruption”. In: MNRAS 480 pp. 1457–1465. doi: 10.1093/mnras/sty1479. arXiv: 1808.00991 [astro-ph.SR].
[21]  Bond, Howard E.;et al, (2003). "An energetic stellar outburst accompanied by circumstellar light echoes". Nature. 422 (6930): 405–408. arXiv:astro-ph/0303513. Bibcode:2003Natur.422..405B. doi:10.1038/nature01508. PMID 12660776.
[22] Green, Gregory M.; et al. (2015) A Three-dimensional Map of Milky Way Dust. The Astrophysical Journal, Volume 810, Issue 1, article id. 25, 23 pp.
[23] Reiman, D & Gohre B (2018), “Deblending galaxy superpositions with branched GANs”, https://ui.adsabs.harvard.edu/abs/2019MNRAS.485.2617R/abstract
[24] Prieto, J. L., Rest, A., Bianco, F. B., Matheson, T., Smith, N., Walborn, N. R., Hsiao, E. Y., Chornock, R., Paredes Álvarez, L., Campillay, A., Contreras, C., González, C., James, D., Knapp, G. R., Kunder, A., Margheim, S., Morrell, N., Phillips, M. M., Smith, R. C., Welch, D. L., & Zenteno, A. (2014), Light Echoes from η Carinae's Great Eruption: Spectrophotometric Evolution and the Rapid Formation of Nitrogen-rich Molecules, \apjl, 787, L8.

