% NSF proposal generation template style file.
% based on latex stylefiles written by Stefan Llewellyn Smith and
% Sarah Gille, with contributions from other collaborators.
%
\documentclass{proposalnsf}

% See this file for a set of pre-defined journal abbreviations
\newcommand{\jas}{{\it J. Atmos. Sci.}}
\newcommand{\jpo}{{\it J. Phys. Oceanogr.}}
\newcommand{\JPO}{{\it J. Phys. Oceanogr.}}
\newcommand{\jfm}{{\it J. Fluid Mech.}}
\newcommand{\jgr}{{\it J. Geophys. Res.}}
\newcommand{\JGR}{{\it J. Geophys. Res.}}
\newcommand{\jmr}{{\it J. Mar. Res.}}
\newcommand{\arfm}{{\it Ann. Rev. Fluid Mech.}}
\newcommand{\dsr}{{\it Deep-Sea Res.}}
\newcommand{\dao}{{\it Dyn. Atmos. Oceans}}
\newcommand{\jam}{{\it Journal of Applied Meteorology}}
\newcommand{\phfl}{{\it Phys. Fluids}}
\newcommand{\phfla}{{\it Phys. Fluids A}}
\newcommand{\PhilTrans}{{\it Philosophical Transactions of the Royal Society, London}}
\newcommand{\gafd}{{\it Geophys. Astrophys. Fluid Dyn.}}
\newcommand{\gfd}{{\it Geophys. Fluid Dyn.}}
\newcommand{\PCE}{{\it Physics and Chemistry of the Earth}}
\newcommand{\PRL}{{\it Physical Review Letters}}
\newcommand{\ProgOc}{{\it Prog. Oceanography}}
\newcommand{\WHOITR}{Woods Hole Oceanographic Institution Technical Report, WHOI-} 
\usepackage{hyperref}
\usepackage{xcolor}

\newcommand{\keyword}[2]

\newcommand{\degrees}{$\!\!$\char23$\!$}
\newcommand{\armin}[1]{\colorbox{cyan}{#1}}
\newcommand{\austin}[1]{\colorbox{yellow}{#1}}
\newcommand{\greg}[1]{\colorbox{orange}{#1}}
\newcommand{\josh}[1]{\colorbox{green}{#1}}
\newcommand{\red}[1]{\colorbox{red}{#1}}
\newcommand{\ana}[1]{\colorbox{red}{#1}}
\newcommand{\changeit}[1]{\colorbox{magenta}{#1}}

\renewcommand{\refname}{\centerline{References cited}}

% This handles hanging indents for publications
\def\rrr#1\\{\par
\medskip\hbox{\vbox{\parindent=2em\hsize=6.12in
\hangindent=4em\hangafter=1#1}}}

\def\baselinestretch{1}

\begin{document}

\begin{center}
{\Large{\bf Project Summary}}\\*[3mm]
{\bf AI detecting reflections in the LSST data} \\*[3mm]

Federica B. Bianco, PI \\
Austin J. Brockmeier, CoI\\
Armin Rest, Collaborator\\
Gregory Dobler, COllaborator\\
Joshua Peek, Collaborator
\end{center}

{\LARGE{
\noindent
\armin{Armin please review}\\
\austin{Austin please review}\\
\greg{Greg please review}\\
\josh{Josh please review}\\
\ana{Ana please review}\\
\changeit{To be changed}}}

\medskip
\medskip
\medskip

\changeit{This is a  proposal to do .....}

\medskip
\noindent
{\bf Intellectual Merit}

The Large Synoptic Survey Telescope (LSST) is the US flagship astronomical project of the 2020s.  LSST will conduct a 10-year survey generating an unprecedented amount of information-dense data: 20Tb/night, every night, for 10 years, pushing the envelope of big-data and data-science.  Observing the whole southern sky every few days, the LSST data reduction software will detect tens of thousands of changes in the sky every night, from transiting planets to exploding stars, but it will entirely miss light echoes: the reflection of ancient stellar explosions that light up the cosmic dust.  The detection of light echoes enables the reconstruction of the history of explosion of stars in our Galaxy, including the potential to detect an unknown galactic supernova, and the tracing of the Milky Way dust structure, which informs the evolution history of the Galaxy and improve inferences on extragalactic sources by constraining extinction. However, light echoes are faint, rare, and diffuse features that change over time, and are hard to detect.  Presently, no automated pipeline for this task exists. LSST will observe the entire southern sky at $\sim$day time intervals and with high sensitivity and spatial resolution, making it the ideal survey to detect light echoes.  This program will support the development of the first pipeline for the automated detection of light echoes by leveraging cutting edge artificial intelligence techniques and its deployment on the LSST computing platform. 
\medskip

\noindent
{\bf Broader Impacts}
\medskip
\changeit{TO DO}
\renewcommand{\thepage} {B--\arabic{page}}

%\newpage

% reset page numbering to 1.  This is helpful, since the text can only
% be 15 pages, and reviewers will want to believe we've kept within
% those limits

\pagenumbering{arabic}
\renewcommand{\thepage} {D--\arabic{page}}

\newpage

\centerline{\bf Results from Prior NSF Support}

\noindent
{\bf Previous Award Title}
{\it award number} (PI); dates, \$amount

Research carried out ....

\ \\

\noindent{\Large \bf PROJECT DESCRIPTION}

\section{Introduction}


About a century ago a true paradigm shift happened in astronomy.  The introduction of digital equipment enabled the discovery and study of the “transients and variable sky” : the sky that in all human mythologies was assumed to be never-changing came to life with the detection of energetic phenomena happening every instant: 10 stars explode every second [a], stars merge deforming space-time in gravitational waves [a], they erupt [a], and more.  A century ago seeing an astronomical transient was a once-in-a-lifetime event, yet today we have barely scratched the surface. Until now, astronomical surveys were limited in either depth, only enabling the observation of the closest and brightest transients, area, observing small regions of sky, or resolution.  Today we can detect ~100 transients each night [4].  LSST will detect tens of thousands, representing not an incremental, but a transformational change. 

{\bf The LSST project} includes the construction of a telescope and a 10-year survey to begin in 2022 [5]. LSST will deliver an unprecedented volume of information-dense data (20Tb/night) with the potential to revolutionize nearly all astrophysics domains [6], from revealing the smallest bodies in the Solar System, to measuring the rate of expansion of the Universe.  LSST will photograph the sky at resolution and depth comparable to those of the Hubble Space Telescope, but unlike previous surveys, limited to small regions, it will deliver observations at a few days’ cadence over the whole southern sky.  This will truly open a window into the transient Universe. 

{\bf Light Echoes} (LEs) are the reflection of stellar explosions on interstellar dust. They appear in the sky as diffuse transients: faint, extended, time-varying [7]. 
As part of its federally funded operations, LSST will produce $10^6$ nightly alerts, each one announcing a changing or moving source (stars, asteroids, etc).  It will detect thousands of transients and variable “point sources” in each image but these diffuse features will be entirely missed.  LEs help characterize stellar explosions by offering the view of an explosion from different lines of sight, a unique opportunity in astrophysics [8,9] 
 and allowing us to revisit events even when the transient was originally not detected [10]. 
{\bf\emph{ The detection of light echoes across a large portion of the sky enables the reconstruction of the history of stellar explosions of our Galaxy, the potential to detect an unknown galactic supernova, and the tracing of the Milky Way dust structure, which will inform the evolutionary history of the Galaxy and improve inference on extragalactic sources by constraining extinction. }}


{\bf However, light echoes are extremely hard to detect}, and, to date, they are still discovered by visual inspection, a method that obviously does not scale to the LSST data volume.  Even crowd-sourcing cannot help this science in the LSST era: simple scaling from the Galaxy Zoo \ana{[CITE]} project indicates that the entire population of the Earth would be insufficient to study the full dataset using the same methods.  We propose to leverage Artificial Intelligence to address this low signal-to-noise regime computer-vision challenge and create the first pipeline for automated detection of light echoes.  {\bf \emph{This will allow us to use images from the LSST survey to produce a census of light echoes that would constrain the history of Milky Way stellar explosions.  The two-year timeline of this proposal aligns perfectly with LSST, with the first commissioning data expected in 2021 and the survey begin observing in late 2022}}.



%%% 
{\LARGE{THE REST OF THIS SECTION  IS STILL BEING DRAFTED AND NOT READY FOR REVIEW}}


\section{Light echoes phenomenology and significance}
Light echoes are the reflections light from astrophysical trasients off interstellar dust in their surroundings.  As the light from a transient propagates into space, it may be reflected towards Earth, and our telescopes, if it encounters a sheet of interstellar dust. The geometry of light echoes is straightforward, a transient and the Earth are at two focal points of a 3D ellipsoid; light from the transient reflected off dust that intersects the same ellipsoid surface reaches Earth at the same time. As time goes by, the ellipsoidal surface expands. Due to the extra travel time, a light echo reaches the observer at a later time than the light directly detected from its source.

Light echoes appear as faint, extended, time-changing features in the sky. The complexity of their shape is inferred by the complexity of the underlying dust structure, while the time-changing aspect is due to the traveling of light across the dust sheet.


Light echoes enable the study of transients from multiple lines of sight, a unique opportunity in astrophysics, allowing insight into the physical mechanism of stellar eruptions and explosions [3,8,9,19,20].  Transients can be studied in their light echoes: this technique has enabled the characterization of historical supernovae with modern instrumentation [8,19,20]; the study of the dust structure that reflects the light allows us to characterize and study stellar eruptions through the characteristics of the ejection of material [21].  Light echoes may reveal unobserved Galactic supernovae.  Furthermore, with LSST we expect to be able to do something unprecedented: build a census of Galactic light echoes that will allow us to reconstruct the history of explosions and eruptions of the Milky Way, providing insight into the evolution of our Galaxy.  Due to its sensitivity, LSST will for the first time enable observation of very faint phenomena on all-sky scale.  Since all Galactic dust shines in reflected light, however faintly, the detection of light echoes will also enable the study of the Galaxy dust structure on large scale.  In combination with studies of the structure from stellar extinction (its ability to absorb light [22]) this will provide an unprecedentedly detailed, three-dimensional map of the Galactic dust which will inform studies of Galaxy formation and improve the characterization of any observed extragalactic source. 


\changeit{difference imaging}
The vast majority of astronomical transients, however, are “point sources”, and detection software, including the LSST pipelines, are optimized for the detection of these point-source transients.



In addition, light echoes are rare, or rather: light echoes that are bright enough to be detected by present surveys, that are not as sensitive, or not as broad as LSST, are rare. LSST will usher a new era in the study of light echoes by observing the entire hemisphere sky repeatedly at high spatial resolution and high sensitivity (limiting single-image depth magnitude g~24). 
The detection of light echoes is a difficult computer vision problem.  The challenges include:
- these features have diverse time-changing shapes, depending on the underlying dust structure and on the travel speed of the light across the dust (dominated by line-of-sight effects)
- reflections inside the telescope constitute a significant source of false positives, which cannot be disambiguated within a single image and requires time sequences to be analyzed.
- light echoes appear as faint and diffuse features, ranging in luminosity until they blend into the image noise.  This problem is distinctly different from most image detection and recognition tasks, where the presence of the target object is unambiguous, and resolution is the most likely limiting factor.  
- the training set is limited: a few hundred examples.  This is a particular challenge in machine learning and artificial intelligence approaches. 

This proposal will support the development of an Artificial Intelligence (AI) model, a Generative Adversarial Neural Network [15], to detect light echoes.  Deep Neural Networks and AI have recently been employed to solve astronomical tasks, primarily due to the rapid increase in data volume, which will culminate with the LSST survey and which requires scalable models that can efficiently extract information from Terabytes of data [e.g. 16].  Over the course of the next two years we expect to develop a model to detect light echoes automatically in sky images and test the model on LSST precursor surveys.  Attempts to develop software to automatically detect light echoes that are based on human-engineered features in images have not been successful, and by far the best method to detect light echoes to date is human-inspection. However, human-inspection obviously cannot scale to the volume of data generated by LSST and to the scope of the problem we ultimate propose to approach: the creation of a comprehensive census of Galactic light echoes to study the history of stellar explosions and eruptions in the Milky Way.  
The light echo detection problem is similar to the well-known, and notoriously difficult problem of detecting smoke-stacks and plumes, which PI Bianco also worked on in the context of Urban Science with a deep learning approach similar to the one proposed here (at the Urban Observatory [17]).  This work will constitute a core portion of the PhD Thesis of one of PI Bianco’s graduate students at the DPA.  We will start from the limited size sample of real light-echo examples, a dataset of simulated light echoes generated analytically, and images of the underlying dust wherever available (infrared observations of the sky reveal dust features, but high-resolution images are not available across the entire sky).  We will use successful Generative Adversarial Neural Networks designed to work under similar circumstances of limited-size real examples and simulated data [18].  Our team will retrain this network on light echo images from telescopes systems similar to the LSST (e.g. from the Blanco telescope Dark Energy Camera, DES, images obtained in proposals co-authored by PI Bianco), and change the network architecture to suit our needs. We expect to be able to test our model on real data as soon as commissioning data from LSST will become available.  In the meantime, our software can run on Blanco/DES images, that share many of the crucial characteristics of the LSST data but are not as sensitive to faint magnitudes, leading to a lower detection rate, looking for known and unknown light echoes.


The two-year timeline of this proposal aligns perfectly with LSST, whose  first commissioning data are expected in 2021 and which will begin observing in 2022.  The work to be conducted under this proposal will produce important software that will allow us to “hit the ground running” and begin detecting light echoes in LSST images as soon as the LSST survey begins to produce data. 

More text.

\subesction{Transients}

More text.
\subsection{Dust}

More text.

\section{Light echoes forward modeling}\label{sec:fwm}

We have developed a method that can create realistic, data-driven light echos as a training set for our neural networks. The method uses real ISM data from 21-cm, neutral hydrogen (HI)  data cubes. 21-cm is a powerful tool here, as we know that HI is well mixed with dust. Further, in many cases velocity is a good proxy for dust (see Tchernyshov & Peek 2017 for details). Finally we now have very high spatial resolution and spatial dynamic range data cubes available (Peek+ 2018, McClure-Griffiths+ 2015) so we can emulate the high resolution imaging of PS1 (?)

If we consider a voxel in the HI data cube to be some intensity at a position ra ($\alpha$), dec ($\delta$), radial velocity (v), we can define a plane such that 

$$1~=~A~*~\alpha~+~B~*~\delta~+~C~*~v$$

We then take all the pixels near this plane in the HI data cube and integrate them along the velocity dimension to create a plane-of-sky image, which we can scale to realistic light-echo amplitudes. If we tip the plane at the right angle to velocity (i.e. set the C-value appropriately), we can mimic the way the light echo slice through distance space and create very realistic structures. The data volume of these HI cubes is so large, and the number of possible values of A and B so variable, we can generate far more synthetic echoes than even the largest networks will need. The first time collaborator peek showed this effect to collaborator ( and expert  light-echo hunter) Rest, Rest was quite astonished with the likeness of the results to real light echoes. He was unsure whether he could distinguish these from real echoes when co-added to real imaging data.


\section{GANS for image processing and analysis}

More text.

\section{Proposed methodology}

More text.
\section{GANS with real and simulated images}

More text.

\section{GANS with real and dust images}

More text.

\section{A model to extract dust structure from light echoes}


More text.

\section{Another Section}

More text.


\section{Time Line and Management Plan}

\subsection{Our team}
PI Bianco is a new faculty member at the University of Delaware (UD) Physics and Astronomy Department and Biden School of Public Policy and Administration. as well as a Resident Faculty member of the new UD Data Science Institute.

PI Bianco’s scientific expertise range from data science to data-driven inference in astronomical transients, from nearby Solar System [12] to distant Supernova explosions [13].  She works closely with the team building and scoping the LSST telescope and survey and represents the community working on LSST in various settings (e.g. recently at the National Academy of Science Space Science Week, invited by the Committee on Astronomy and Astrophysics). 

Among the many transient phenomena in PI Bianco’s research portfolio, she has worked on LEs collaborating with CoI Rest leading the data collection and analysis at the Las Cumbres Observatory Global network, which contributed the majority of the image data analyzed in [3] and focusing on the extraction of lightcurves from LEs that supports the analysis of typing and spectral evolution [19, 20, 24]
and studies that relate diversity in transients to asymmetry as assessed from light echoes. Additionally, she is working on technically similar computer vision issues in the context of urban imaging at the Urban Observatory \changeit[CITE], a multi-city facility lead by Collaborator Dobler where urban environments are studied through imaging and techniquest adapted from astronomy (i.e. photometry, spectroscopy) developing methodology to detect of faint transient diffused feature generated by building HVAC system and other emissions \changeit[CITE].

\armin{PLEASE REVIEW}
Rest is a world-renowned expert in light echoes, from detection to transient characterization. He has lead several succesfull programs devoted to the study of light echoes, including current NSF program \colorbox{red}{XXX}. He also has lead the acquisition and analysis of light echo data used in this proposal \colorbox{red}{NOAO PROPOSAL NUMBERS FOR DECAM IMAGES}. In this project, he will supervise a Graduate student from John Hopkins University in year 1, as well as co-advising the UD Graduate students in developing a detailed work plan and set up the data access, ingestion, and manual labeling to create a standardized dataset of light echoes over which the performance of our models will be measured.   

\austin{PLEASE REVIEW}

CoI Brockmeier is a new faculty member at the University of Delaware (UD) Electrical and Computer Engineering and,  as well, a Resident Faculty member of the new UD Data Science Institute. His expertise in signal processing range from hyperspectral to 

\section{Summary:  Significance of proposed work}

More text.

\subsection{Intellectual Merit}

More text.
\section{Unfunded Collaborators}

\subsection{CoI Armin Rest, STScI}




\subsection{Collaborator Joshua Peek, STScI}
Peek is an Associate Astronomer and the Data Science Mission Office Project Scientist at the Space Telescope Science Institute. He developed the forward modeling methodology for light echoes and produced several simulations as described in~\autoref{sec:fwm}. He will make his existing simulations available and coach students in the methodology to create a comprehensive training sample to be fed to our GANs. 

\subsection{Collaborator Gregory Dobler, UDelaware}
Dobler is an Assistant Professor at the Department of Physics and Astronomy, Biden School of Public Policy and Administration, and Data Science Institute at the University of Delaware.  He has extensive experience modeling interstellar dust emission, absorption properties, and grain composition as well as the application of machine learning (including neural networks and deep learning algorithms) to ~PB-scale imaging data sets.  In this project he will participate in the effort to incorporate dust properties into the analysis and applying machine learning algorithms to simulated data, as well as co-supervising the graduate and undergraduate students.

\subsection{Broader Impacts}
The project will involve a University of Delaware graduate student, and it will constitute the core of their thesis.  We will collaborate closely with a Johns Hopkins graduate student in the first year of the project whose thesis is also on light echoes and transient identification in light echoes.  We will involve one undergraduate each year on the project. 

The broader impact of this proposal will focus on machine learning and data science education of undergraduate and graduate students at the University of Delaware (UD) and John Hopkins University. 
This skill are becoming indispensable in an increasingly technological world.  As part of her faculty responsibility toward the Department of Physics and Astronomy (DPA) at the UD, PI Bianco is developing data science and machine learning curriculum for physicists and astronomers (two classes have been developed this year: Data Science for Physical Scientists\footnote{\url{https://github.com/fedhere/dsps}} which is offered at the undergraduate and graduate level, and Machine Learning for Time Series Analysis which will be offered at the graduate level next semester). 
The UD graduate student and the undergraduates involved in this proposal will be encouraged to take these classes, which are project based designed to provide a hands-on learning environment ideal for developing data-driven research skills. 

Projects related to this proposal will offer a research opportunity for undergraduate students that are interested in the new 4+1 Physics and Data Science program at UD. UD launched a Master Degree in Data Science\footnote{https://www.msds.udel.edu/} in 2018 and will expand it to include a 4+1 joint degree with the DPA next year (the 4+1 is under academic review).

Another important opportunity  will be available to all student involved in this proposal (at UD and Hopkins), and to all UD students, through a UD hackathon program that PI Bianco and CoI Brockmeier will start at the UD Data Science Institute.  This program will organize UD-wide hackathons for all UD students at least twice a year.

Hackathon and data dives are prime examples of active learning activities.  We will organize hackathons twice yearly as a weekend long event.  Students will work in self-assembled groups on projects of their choice and produce a deliverable, which they will present at the end of the event in front of judges and a small audience.  Open to all interested undergraduates, the hackathons will have a variety of topics, both proposed by the UD faculty and by the students themselves (LSST-related projects will be made available at each session, including projects related to the research supported by this proposal). This is an ideal venue for building superskills, especially collaborative and communication skills, in our students. LSST has supported hackathons in various environment as a way to jump-start LSST related research and PI Bianco has been involved in the leadership and organization of many of these events (e.g. http://fbb.space/LSSTHackathonCCA/WPgallery.html, https://lsst-tvssc.github.io/TVS-SMWLV2019/).

A typical hackathon event starts with project pitches.  Students are encouraged to present project ideas, but projects are also available and presented by the moderating faculty and postdoc.  Occasionally, external agencies or companies are invited to pitch projects as well.  In that case, they may provide prizes to the winning teams.  A pitch would include a problem statement, the desired deliverable, and the skills required for the project.
Groups self-assemble to work on projects based on interest and required expertise and work continuously, including on the preparation of a 5-10 minutes slides presentation that concludes the event.  During the hackathon, mentors are available to help and advise the students on methodology, and a moderator makes sure the tight timeline is respected, including containing the initial pitches and team assembling time, and assuring that students prepare the slides timely for the presentation. 
Project results are presented by each team.  Feedback is given separately on the presentation, by an assigned "critique", and on the content by the judges, that have priority in asking questions and giving feedback, then time permitting additional questions are asked by the audience.  Projects are ranked based on their success in four main areas: 

- the quality of the presentation

- the quality of the deliverable

- how well the deliverable complies with the request formulated in the pitch 

- the application of adequate and appropriate methodologies.

In these events, the students see the full life-cycle of a data-driven project, from conception to presentation.  This is a rare opportunity for them since in course-based instruction, even in project-based classes, generally the research is guided and extended over time, often resulting in a dispersive, as opposed to immersive, experience. The postdoc hired under this project will serve as a mentor and organizer, together with PI and CoI Bianco and Brockmeier, and as a judge at the final presentation showdown (see Postdoc Mentoring Plan)


\clearpage
\noindent{\Large \bf j Supplementary Material}
\clearpage
\section{Postdoctoral Mentoring Plan - 1 page}

This proposal will support 50\% of the effort of a postdoctoral researcher.  The postdoctoral researcher mentoring plan for this program will be overseen by the PI Dr. Bianco (Assistant Professor at the UD, Physics and Astronomy - DPA) and CoI Dr. Brockmeier (Assistant Professor at the UD, Electrical Engineering).  The different domains of Prof.s Bianco and Brockmeier will provide a complete multi-faceted point of view in reviewing and advising the postdocs on science and job-success related matters. 
The postdoctoral researcher will be recruited with a solicitation issued by PI Bianco on November 2019.  The PIs will create a Postdoctoral Selection Committee including CoI Brockmeier and two senior faculty members within the Department of Physica and Astronomy at UD to review the applications.

To assure that a diverse pull of candidates is reached, PI Bianco has has posted the job announcement on several venues that target diverse scholars (e.g. Diverse Issues in Higher Ed. and Hispanic Outlook).  If a  scholar from an under-represented minority is hired, they will benefit from the commitment of UD and specifically the renewed commitment of the DPA to create an equitable and supportive environment with a new Diversity and Inclusion Committee\footnote{http://bit.ly/UDDPADIcommittee}, created and chaired by PI Bianco at the UD DPA.

The PI and CoI will offer, as part of the mentoring program: 
1) Training in preparation of grant proposals: Being able to effectively and consistently find funding is a compulsory skill required of successful researchers.  The postdoctoral researcher should be coached in and encouraged to submit proposals (e.g. postdoctoral fellowship, HST proposals) and encouraged to serve on at least one review panel to gain a deeper understanding of the review process. 
2) Publications and presentations: The mentors will support (through this grant and her start-up funds) travel support to assure the postdocs can present their work at conferences and publish in high impact journals.  The postdocs should travel to the LSST annual Community Workshop, international LSST meetings (e.g. LSST@Europe, on a bi-annual basis), and topical AI meetings. 
3) Mentoring: The postdoc will have opportunities to interact with and mentor students. They will mentor the graduate and undergraduate students working with PI Bianco and CoI Rest, but the PI will also involve the postdoc in hackathons as mentor, both in LSST meetings and at the UD Data Science Institute, where PI Bianco and CoI Brockmeier are developing a hackathon program of 2 events each year (PI Bianco has experience developing such programs at NYU Center for Urban Science and Progress and elsewhere, see biosketch).  The postdoctoral researcher will be involved in  planning and preparing the hackathons and will serve as mentors during the hackathon sessions and as judges at the hackathon presentations.
4) Career Counselling: A likely outcome for the postdoc is  employment outside of academia: in recent years the number of graduating PhD physics students ranged between 1500 and 2000, while the number of  faculty hires ranges between 200 and 300\footnote{\url{https://www.aip.org/sites/default/files/statistics/physics-trends/fall16-phdsubfield-p1.pdf}}\footnote{\url{https://www.aip.org/statistics/physics-trends/number-faculty-hired-physics-departments}}.  The postdoctoral researcher will data science learning skills that are crucial in the tech job market and will be encouraged to continually discuss their career options with their mentors, peers, and collaborators, so that they can successfully gauge the appropriate skills required for their chosen career path.
Assessment: The personal progress of the postdoctoral researchers will be measured against the expectations laid out in this mentoring plan, i.e.: number of papers and proposals submitted, the number of seminars given, quality of lectures, and  effectiveness of their mentoring (student success).  Every 4 months the postdoctoral research will meet separately with PI Bianco and CoI Brockmeier so that they will be able to raise concerns about mentoring.  Dr. Bianco and Dr. Brockmeier will share a report with the postdoctoral research every 4 months.  


\clearpage


\newpage
\pagenumbering{arabic}
\renewcommand{\thepage} {E--\arabic{page}}

\bibliography{draft}
\bibliographystyle{jponew}

\newpage
\pagenumbering{arabic}
\renewcommand{\thepage} {G--\arabic{page}}
\noindent{\Large \bf BUDGET JUSTIFICATION}

\end{document}
